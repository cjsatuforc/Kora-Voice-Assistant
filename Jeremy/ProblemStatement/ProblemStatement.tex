\documentclass[onecolumn, draftclsnofoot,10pt, compsoc]{IEEEtran}
\usepackage{graphicx}
\usepackage{url}
\usepackage{setspace}
\usepackage{url}
\usepackage{geometry}
\geometry{textheight=9.5in, textwidth=7in}
\setlength\parindent{0pt}

% 1. Fill in these details
%\def \CapstoneTeamName{		The Cleverly Named Team}
\def \CapstoneTeamNumber{		8}
\def \GroupMemberOne{			Jeremy Fischer}
%\def \GroupMemberTwo{			Thomas Kuhn}
%\def \GroupMemberThree{			Karl Popper}
\def \CapstoneProjectName{		NLP For Digital Manufacturing}
\def \CapstoneSponsorCompany{	Autodesk}
\def \CapstoneSponsorPerson{		Patti Vrobel}

% 2. Uncomment the appropriate line below so that the document type works
\def \DocType{		Problem Statement
				%Requirements Document
				%Technology Review
				%Design Document
				%Progress Report
				}
			
\newcommand{\NameSigPair}[1]{\par
\makebox[2.75in][r]{#1} \hfil 	\makebox[3.25in]{\makebox[2.25in]{\hrulefill} \hfill		\makebox[.75in]{\hrulefill}}
\par\vspace{-12pt} \textit{\tiny\noindent
\makebox[2.75in]{} \hfil		\makebox[3.25in]{\makebox[2.25in][r]{Signature} \hfill	\makebox[.75in][r]{Date}}}}
% 3. If the document is not to be signed, uncomment the RENEWcommand below
\renewcommand{\NameSigPair}[1]{#1}

%%%%%%%%%%%%%%%%%%%%%%%%%%%%%%%%%%%%%%%
\begin{document}
\begin{titlepage}
    \pagenumbering{gobble}
    \begin{singlespace}
    	\includegraphics[height=4cm]{Images/coe_v_spot1}
        %\hfill 
        % 4. If you have a logo, use this includegraphics command to put it on the coversheet.
        %\includegraphics[height=4cm]{CompanyLogo}   
        \par\vspace{.2in}
        \centering
        \scshape{
            \huge CS Capstone \DocType \par
            {\large\today}\par
            \vspace{.5in}
            \textbf{\Huge\CapstoneProjectName}\par
            \vfill
            {\large Prepared for}\par
            \Huge \CapstoneSponsorCompany\par
            \vspace{5pt}
            {\Large\NameSigPair{\CapstoneSponsorPerson}\par}
            {\large Prepared by }\par
            Group\CapstoneTeamNumber\par
            % 5. comment out the line below this one if you do not wish to name your team
            %\CapstoneTeamName\par 
            \vspace{5pt}
            {\Large
                \NameSigPair{\GroupMemberOne}\par
                %\NameSigPair{\GroupMemberTwo}\par
                %\NameSigPair{\GroupMemberThree}\par
            }
            \vspace{20pt}
        }
        \begin{abstract}
        % 6. Fill in your abstract    
        	The \textit{NLP For Digital Manufacturing} project is a proof of concept project that will integrate an open source natural language processing library into an Autodesk 3-D computer aided design (CAD) software product. The goal is to create a framework for a manufacturing bot (M-bot) that can execute voice commands such as “rotate the design 90 degrees to the left”, and interview the designer on decisions made like “did you make this facet a sphere for aesthetic reasons or to meet design constraints?” This will allow Autodesk to begin collecting the \textit{why} from designers versus solely the \textit{what} that they have been collecting since their inception. This new addition to their data set will allow for machine learning to transform the computer into a teammate that can suggest design choices. The motivation for this proof of concept is to document useful resources used and major obstacles that Autodesk would need to overcome and implement themselves if they’d like to pursue a fully-fledged M-bot.
        \end{abstract}     
    \end{singlespace}
\end{titlepage}
\newpage
\pagenumbering{arabic}
\tableofcontents
% 7. uncomment this (if applicable). Consider adding a page break.
%\listoffigures
%\listoftables
\clearpage

% 8. now you write!
\section{Problem}
Since the inception of publically available computers until around 2006 society has interacted with computers in the same manner – keyboard and mouse. Recently there’s been a moving trend toward making computers and machines more interactive. Nintendo launched the Wii in 2006 and Microsoft’s XBOX division followed in 2010 with the Kinect \cite{wii}\cite{kinect}. Apple launched Siri in 2010 and Microsoft and Google followed shortly after with their smart assistants \cite{siri}. Other companies are starting to see the competitive advantage of having their products more interactive.
\\ \\
Autodesk realizes this and has started experimenting with making designs created by their software interactive in augmented and virtual reality. They now want to dip their toes into natural language processing. However, Autodesk has the opportunity to gain much more from making their products more interactive than simply retaining and attracting users due to the new ease of use. From a business perspective, enabling natural language processing allows for a new addition to user data that is rich in potential!

\section{Solution}
Autodesk has been collecting user data since their first product – AutoCAD. But this data merely contains \textit{what} the user designed and \textit{what} decisions the user made. With the integration of a Manufacturing bot (M-bot) that supports natural language processing, Autodesk’s products would be able to collect the \textit{why}. Why the user made a decision. For example, did the user make a particular facet in the design a sphere for aesthetic reasons or because it was the only shape that met the design’s constraints? The M-bot would be able to interview the user to collect this type of information. After enough data is collected Autodesk would be able to run machine learning algorithms on top of it and be able to offer suggestions to the designer making M-bot even more powerful. Besides an interviewing capability, M-bot would be able to execute voice commands. Commands such as “rotate the design 90 degrees to the left”, and “save the design as \textit{bicycle\_iteration\_one}.” Both of these capabilities transform the computer from a tool to a teammate that can take on tasks and help suggest changes.
\\ \\ 
M-bot won’t be to this level for quite some time, but Autodesk has to start somewhere. That’s where the NLP For Digital Manufacturing capstone project comes in. This project will be in charge of laying out the natural language processing framework and answering proof of concept questions like: what open source libraries can be used? What big obstacles must Autodesk develop themselves? How can M-bot interview the user without being obnoxious? How will this data be stored? And how do we integrate this into an Autodesk Product?

\section{Deliverables}
For the purposes of this project, our deliverables will be mostly learning material that Autodesk can use to pursue M-bot in the future, along with an M-bot prototype. We will deliver\dots
\begin{itemize}
	\item A list of fully developed natural language processing open source libraries that can be used to build off of
	\item A document that indicates obstacles that Autodesk would have to overcome and develop to pursue M-bot’s future
	\item A high-level workflow of how M-bot will interview and collect the why from the designer
	\item An open source natural language processing library integrated into one of Autodesk’s products. Which product is unknown right now, but most likely Fusion 360 or one with a rich application programming interface (API) set
	\item M-bot being able to execute the command “rotate the design 90 degrees to the left”
	\item M-bot asking the interview question “Did you do this for an aesthetic reason or a design constraint reason?” and storying the question paired with the user’s answer in a file
\end{itemize}

\bibliography{problemStatementBib} 
\bibliographystyle{ieeetr}

\end{document}