\documentclass[10pt, draftclsnofoot, onecolumn]{IEEEtran}
\usepackage[utf8]{inputenc}
\usepackage[letterpaper, margin=0.75in]{geometry}
\include{pygments.tex}
\title{}
\author{Austin Row \\
CS 461 \\
NLP for Digital Manufacturing-Problem Statement \\
Fall 2017 \\}
\date{October 2017}

\begin{document}

\maketitle

Our project consists of implementing a voice interface for a 3D modelling product. Human-computer interaction is a field that has been developing and progressing since the earliest computers and the current popular mediums of the mouse and keyboard do not represent the most natural ways for humans to interface with computers. This project proposes to explore the potential of a voice interface as it relates to the future of interacting with computers by implementing such an interface with Autodesk’s Fusion 360 3D modelling product. The interface will listen for verbal commands from the user and will perform them in a timely manner as defined by standard user-experience principles. This will offer the user a more natural experience which, as a result, will increase the speed and ease with which the user can accomplish their goals within the product.

\clearpage

\section*{Problem Description}
Since the development of the first means of computing, the methods by which individuals have interacted with computers have been evolving to fit the needs and desires of users. Some of the earliest methods for interacting with what could technically be called computers were through punch cards and slightly later via keyboards. Eventually keyboards came into the form that most people are now familiar with and this was followed by the first mouse for a computer and even joysticks for video games. More recently human computer interaction has attempted to venture into more natural realms with the development of modern motion tracking technology for gaming such as Nintendo’s Wii remotes and Microsoft’s Kinect for Xbox. The constant evolution of the way that people interact with technology illuminates an inherent issue: the methods that exist for interfacing with computers and associated technology are not yet optimal. There is an incessant push to make the user experience feel natural and each of the methods that have been released and subsequently widely adopted in the past have pushed communication with computers closer to this goal, but it is not a goal that has been fully realized yet. \\ 

With the user experience experience in mind, what we hope to do with this project is to explore a newer way to interface with the computer. Specifically, we look to explore how a voice interface can be used to improve the user experience. \\

The project will be a case study of how a voice interface can impact the user experience in the context of using Autodesk’s 3D modelling software Fusion 360. Currently the methods for interacting with Fusion 360 extend only to the mouse and keyboard. It is our and our client’s belief that there exists interfaces with the potential to make the user experience more fluid and to reduce the hindrance of having to use the mouse and keyboard. Ultimately the problem that we will address is the fluidity of the user experience and how it is limited by the traditional computer interaction methods of the mouse and keyboard. \\

\section*{Proposed Solution}
Our proposal is composed of two parts. The first is to improve upon the traditional interface with the computer by exploring existing methods for, and subsequently implementing, a proof-of-concept voice interface for Autodesk’s Fusion 360 product. Speech is a natural mode of communication for humans and we believe that it has the potential to drastically reduce the time it takes for a user to achieve their end goal within the product if it can be leveraged into a robust interface with Fusion 360. To achieve this goal we will explore existing open source tools and utilize the expansive Fusion 360 API. \\ 

The second part of our proposal is an additional goal that will not be considered necessary for the completion of a successful project, but will be attempted upon the completion of what we consider a “successful” implementation of the Fusion 360 voice interface. Upon completion of the voice interface, we will look to expand upon the potential inherent in a voice interface by applying machine learning techniques to build an intelligent agent that looks to assist the user by offering verbal suggestions and notices regarding projects that they are working on. \\

\section*{Performance Metrics}
Our performance on this project will be judged on primarily on one point. The completion of a voice interface for Autodesk’s Fusion 360 that responds to basic user commands within time limits defined by standard user-experience principles will be judged as a successful completion of the project. Unsuccessful implementation of such an interface will be judged as failure to satisfactorily complete the project. Any work done to complete an intelligent agent will be considered ancillary to the project and the completion or lack thereof of any such agent will not affect the judgement of project performance. \\

\end{document}
