\documentclass[onecolumn, draftclsnofoot,10pt, compsoc]{IEEEtran}
\usepackage{graphicx}
\usepackage[section]{placeins}
\usepackage{url}
\usepackage{setspace}
 
\usepackage{alltt}                                           
\usepackage{float}
\usepackage{color}
\usepackage{url}

\usepackage{geometry}
\geometry{textheight=9.5in, textwidth=7in}
\setlength\parindent{0pt}

\usepackage{tabu}
\usepackage{longtable}

\usepackage{xspace}
\usepackage{pgfgantt}
\usepackage{subcaption}

% 1. Fill in these details
\def \CapstoneTeamName{\textbf{Insert Team Name Here} }
\def \CapstoneTeamNumber{8}
\def \GroupMemberOne{James Stallkamp}
\def \GroupMemberTwo{Jeremy Fischer}
\def \GroupMemberThree{Austin Row}
\def \CapstoneProjectName{Kora}
\def \CapstoneSponsorCompany{Autodesk}
\def \CapstoneSponsorPerson{Patti Vrobel}
\def \botname{Kora\xspace}

% 2. Uncomment the appropriate line below so that the document type works
\def \DocType{		%Problem Statement
				%Requirements Document
				%Technology Review
				%Design Document
				Progress Report
				}
			
\newcommand{\NameSigPair}[1]{\par
\makebox[2.75in][r]{#1} \hfil 	\makebox[3.25in]{\makebox[2.25in]{\hrulefill} \hfill		\makebox[.75in]{\hrulefill}}
\par\vspace{-12pt} \textit{\tiny\noindent
\makebox[2.75in]{} \hfil		\makebox[3.25in]{\makebox[2.25in][r]{Signature} \hfill	\makebox[.75in][r]{Date}}}}
% 3. If the document is not to be signed, uncomment the RENEWcommand below
\renewcommand{\NameSigPair}[1]{#1}

%%%%%%%%%%%%%%%%%%%%%%%%%%%%%%%%%%%%%%%
\begin{document}
\begin{titlepage}
    \pagenumbering{gobble}
    \begin{singlespace}
    	\includegraphics[height=4cm]{coe_v_spot1}
        %\hfill 
        % 4. If you have a logo, use this includegraphics command to put it on the coversheet.
        \par\vspace{.2in}
        \centering
        \scshape{
            \huge CS Capstone \DocType \par
            {\large\today}\par
            \vspace{.5in}
            \textbf{\Huge\CapstoneProjectName}\par
            \vfill
            {\large Prepared for}\par
            \Huge \CapstoneSponsorCompany\par
            \vspace{5pt}
            {\Large\NameSigPair{\CapstoneSponsorPerson}\par}
            {\large Prepared by }\par
            Group\CapstoneTeamNumber\par
            % 5. comment out the line below this one if you do not wish to name your team
            %\CapstoneTeamName\par 
            \vspace{5pt}
            {\Large
                \NameSigPair{\GroupMemberOne}\par
                \NameSigPair{\GroupMemberTwo}\par
                \NameSigPair{\GroupMemberThree}\par
            }
            \vspace{20pt}
        }
        \begin{abstract}
			This document outlines the production progress of Kora in Fall 2017, including obstacles and steps moving forward.
        \end{abstract}     
    \end{singlespace}
\end{titlepage}
\newpage
\pagenumbering{arabic}
\tableofcontents
% 7. uncomment this (if applicable). Consider adding a page break.
%\listoffigures
%\listoftables
\clearpage

% 8. now you write!

\section{Purpose}
	Kora is a proof of concept project that will integrate a natural language processing library into Autodesk's 3-D computer aided design software, Fusion.
	Kora will be a speech-based virtual assistant for Fusion that lets users perform any subset of tasks within the product, such as saving a document or opening a menu, by verbally instructing it to perform the task.
	As a stretch goal, Kora will be capable of questioning the user and using responses to predict and automatically assist with future user behavior.
	
	Kora will offer users a tool that decreases the time required to achieve their goals within Fusion by offering an interface that runs in parallel with and complements the keyboard and mouse.
	If the stretch goal is achieved, Kora will further increase productivity by learning to automate specific workflows within the product.

\section{Goals}
	\subsection{Main Goals}
		\begin{itemize}
			 \item{
			 	Kora will allow the user to perform tasks in Fusion via a voice interface.
				Each voice command given by the user will be mapped to a specific task which will then be performed in the open Fusion design.}
			\item{
				All interactions with Kora will be logged for debugging and future development purposes.}
		\end{itemize}
	\subsection{Stretch Goals}
		\begin{itemize}
			\item{
				Kora will periodically ask the user questions regarding why they have performed specific tasks.
				The questions asked by Kora will pertain to specific predefined contexts.}
			\item{
			Kora will record user responses and save a transcript along with contextual information.}
			\item{
				User responses to interview questions will be used to try to predict and assist with future user actions.}
		\end{itemize}
\section{Project Progress}
	This term the group was introduced to Kora and what Autodesk wants to get out of this project.
	The group wrote the problem statement document which consisted of a high level definition of Kora and a description of the problem Kora will be solving.
	Afterwards, the requirements document was created which outlined "what" Kora should be able to do.
	The requirements document represents the contract between the group and Patti that details what we will produce.
	From there, we wrote the technology review documents, which outline the specific technologies we will use to develop Kora.
	We ended the term by writing the design document, which represents "how" Kora will be developed.
	The design document also serves as a development roadmap for the rest of the year.
	

\section{Obstacles}
	The biggest obstacle this group is facing and will continue to face throughout Kora's development is not having access to the Fusion source code.
	This obstacle constrains Kora's success, because Kora can only provide services which the Fusion API offers.
	With that being said, the services offered by Fusion's API are extensive, so we are hopeful it does not become too big of an issue.
	From what we have seen so far, the API does not provide a clear solution for creating a pop-up.
	This may be a problem we will face when implementing the on-screen Kora widget.
	
	
\section{Retrospective}
		\begin{center}
	\begin{longtabu} to \textwidth {|
			X[2,l]|
			X[5,l]|
			X[5,c]|
			X[5,l]|}
		\hline
		\textbf{Week} & \textbf{Positives} & \textbf{Deltas} & \textbf{Actions} \\ \hline
		
		Week 1 
		& 
		{We found an intro to Natural Language Processing (NLP) video series taught by a Stanford Professor on YouTube.}
		& 
		{We have not met with Patii yet, so we are unsure what the project is about. Thus far we only know NLP is involved.}
		& 
		{Scheduled a 30 minute "Get to know each other" meeting with Patti}
		\\ \hline
		
		Week 2 
		&  
		{We got an overview of what Autodesk does, as well as Patti's role. We touched on the project and what she's hoping to get out of it.}
		&
		{Are meeting wasn't long enough to get a full understanding of the project.}
		&  
		{We scheduled an hour long follow up meeting for Monday October 9th to dive deeper into the project.}  
		\\ \hline
		
		Week 3 
		&
		{We met with Patti for an hour Monday afternoon and got a better understanding of the project.} 
		& 
		{We won't be getting the Fusion source code, we will have to use the Fusion API instead.} 
		&
		{We started examining the breadth and depth of the Fusion API.} 
		\\ \hline
		
		Week 4 
		&
		{Met with our TA, Juneki, and discussed our GitHub repo structure.
		We also met with Patti, and got constructive criticism on our problem statement documents} 
		& 
		{We need to add a Documentation folder in our GitHub repository, and make minor tweaks to our problem statement documents.} 
		& 
		{Added the Documentation folder and added the critiques made by Patti to our problem statements.} 
		\\ \hline
		
		Week 5 
		&
		{We got open source NLP project suggestions from Juneki} 
		& 
		{Our weekly meeting with Patti got canceled because she was traveling.} 
		& 
		{We began researching the project suggestions given by Juneki, and editing our requirements document rough draft.} 
		\\ \hline
		
		Week 6 
		&
		{The group finished the requirements document.} 
		& 
		{There was confusion about whether this project should be considered research based.} 
		& 
		{We went to the class regarding research based projects to determine whether our project should be considered research based.} 
		\\ \hline
		
		Week 7 
		&
		{Jeremy, Austin, and James met and brainstormed the nine technology components of the project. These components were used for the tech review documents.} 
		& 
		{The group came across the obstacle that certain technologies are operating system dependent.}
		& 
		{We ran the nine components by Patti, and began working on our individual technology reviews} 
		\\ \hline
		
		Week 8 
		&
		{We finalized our individual technology reviews.} 
		& 
		{Our weekly meeting with Patti got canceled because she was traveling.} 
		& 
		{The technology review documents will be peer reviewed in class this week, and the edits will be added afterwards.}  
		\\ \hline
		
		Week 9 
		&
		{Thanksgiving Break} 
		& 
		{Thanksgiving Break} 
		& 
		{Thanksgiving Break} 
		\\ \hline
		
		Week 10 
		&
		{The group finished the design document.}
		& 
		{Our weekly meeting with Patti got canceled because she was traveling.} 
		& 
		{We will turn in the design document, progress report, and progress report presentation.}  
		\\ \hline
		
		
	\end{longtabu}
\end{center}

\section{Moving Forward}
	This term consisted of planning and writing documents.
	Now that the group fully understands Kora's purpose and devised a plan for developing Kora, including which technologies and platforms will be used, the next term will consist of the project's actual implementation.
	The group is hopeful that by the end of next term Kora's voice-to-action pipeline will be connected. This means Kora will be able to execute a command such as "save this design as myDesign1."

\end{document}
