\documentclass[onecolumn, draftclsnofoot,10pt, compsoc]{IEEEtran}
\usepackage{graphicx}
\usepackage[section]{placeins}
\usepackage{url}
\usepackage{setspace}
 
\usepackage{alltt}                                           
\usepackage{float}
\usepackage{color}
\usepackage{url}

\usepackage{geometry}
\geometry{textheight=9.5in, textwidth=7in}
\setlength\parindent{0pt}

\usepackage{xspace}
\usepackage{pgfgantt}
\usepackage{subcaption}

% 1. Fill in these details
\def \CapstoneTeamName{\textbf{Insert Team Name Here} }
\def \CapstoneTeamNumber{8}
\def \GroupMemberOne{James Stallkamp}
\def \GroupMemberTwo{Jeremy Fischer}
\def \GroupMemberThree{Austin Row}
\def \CapstoneProjectName{Kora}
\def \CapstoneSponsorCompany{Autodesk}
\def \CapstoneSponsorPerson{Patti Vrobel}
\def \botname{Kora\xspace}

% 2. Uncomment the appropriate line below so that the document type works
\def \DocType{		%Problem Statement
				%Requirements Document
				%Technology Review
				Design Document
				%Progress Report
				}

\newcommand{\designConcernRef}[2][]{
    #2 #1
}
\newcommand{\designElementDef}[4]{
    \subsubsection{#1}
    \begin{tabular}[t]{r p{6in}}
        Type: & #2 \\
        Purpose: & #3 \\
        Author: & #4 \\
    \end{tabular}
}
\newcommand{\designElementRef}[2]{
    \subsubsection{#1}
    \begin{tabular}[t]{r p{6in}}
        See #2 & \\ %#2 should be element identifier (section where it's defined or ID that can be used to find it)
    \end{tabular}
}
\newcommand{\NameSigPair}[1]{\par
\makebox[2.75in][r]{#1} \hfil 	\makebox[3.25in]{\makebox[2.25in]{\hrulefill} \hfill		\makebox[.75in]{\hrulefill}}
\par\vspace{-12pt} \textit{\tiny\noindent
\makebox[2.75in]{} \hfil		\makebox[3.25in]{\makebox[2.25in][r]{Signature} \hfill	\makebox[.75in][r]{Date}}}}
% 3. If the document is not to be signed, uncomment the RENEWcommand below
\renewcommand{\NameSigPair}[1]{#1}

%%%%%%%%%%%%%%%%%%%%%%%%%%%%%%%%%%%%%%%
\begin{document}
\begin{titlepage}
    \pagenumbering{gobble}
    \begin{singlespace}
    	\includegraphics[height=4cm]{coe_v_spot1}
        %\hfill 
        % 4. If you have a logo, use this includegraphics command to put it on the coversheet.
        \par\vspace{.2in}
        \centering
        \scshape{
            \huge CS Capstone \DocType \par
            {\large\today}\par
            \vspace{.5in}
            \textbf{\Huge\CapstoneProjectName}\par
            \vfill
            {\large Prepared for}\par
            \Huge \CapstoneSponsorCompany\par
            \vspace{5pt}
            {\Large\NameSigPair{\CapstoneSponsorPerson}\par}
            {\large Prepared by }\par
            Group\CapstoneTeamNumber\par
            % 5. comment out the line below this one if you do not wish to name your team
            %\CapstoneTeamName\par 
            \vspace{5pt}
            {\Large
                \NameSigPair{\GroupMemberOne}\par
                \NameSigPair{\GroupMemberTwo}\par
                \NameSigPair{\GroupMemberThree}\par
            }
            \vspace{20pt}
        }
        \begin{abstract}

        \end{abstract}     
    \end{singlespace}
\end{titlepage}
\newpage
\pagenumbering{arabic}
\tableofcontents
% 7. uncomment this (if applicable). Consider adding a page break.
%\listoffigures
%\listoftables
\clearpage

% 8. now you write!


\section{Frontspiece}
	\subsection{Date of Issue and Status}
		This document was issued November 30, 2017 and is the first draft of the design document.
	
	\subsection{Issuing Organization}
		This document has been issued by Oregon State University and Autodesk via the senior design capstone class.
	
	\subsection{Authorship}
		Jeremy Fischer, Austin Row, and James Stallkamp are the authors of this document and the developers of Kora.
		
	\subsection{Change History}
		\begin{table}[H]
			\centering
			\caption{Change History}
			\label{my-label}
			\begin{tabular}{|l|l|}
				\hline
				\textbf{Date}     & \textbf{Change Description}   \\ \hline
				November 30, 2017 & {First design document draft} \\ \hline
			\end{tabular}
		\end{table}

\section{Introduction}
	\subsection{Purpose}
		The purpose of this design document is to outline how Kora's workflows will be completed and connected.
		More generally, this document describes how the client's requirements will be met.
		Kora's developers will use this document as a roadmap during implementation.

	\subsection{Scope}
		This document focuses on the relationships between Kora's components and their individual processes, and how they work together to satisfy the project's requirements.
	
    \subsection{Context}
        TODO: FILL THIS IN

	\subsection{Summary}
		Kora will be a speech-based virtual assistant for Fusion that lets users perform any one of a subset of tasks within the product, such as saving a document or opening a menu, by verbally instructing it to perform the task.
		Workflows in Fusion that are not suited for handling by a voice interface will not be supported by Kora.
		As a stretch goal, Kora will be capable of questioning the user and using responses to predict and automatically assist with future user behavior.
		It will be a plugin that is bundled with Fusion and will be part of the product's standard download. 
		
		Kora will offer users a tool that decreases the time required to achieve their goals within Fusion by offering an interface that runs in parallel with and complements the keyboard and mouse.
		If the stretch goal is achieved, Kora will further increase productivity by learning to predict and automate specific workflows within the product.
    
\section{References}


\section{Glossary}
	\begin{table}[H]
		\centering
		\caption{Glossary}
		\label{my-label}
		\begin{tabular}{|l|l|}
			\hline
			\textbf{Term} & \textbf{Definition} \\ \hline
			Kora & The virtual assistant that is the focus of this project \\ \hline
			NLP & Natural Language Processing \\ \hline
			API & Application Programing Interface \\ \hline
			CAD & Computer Aided Design \\ \hline
			CAM & Computer Aided Manufacturing \\ \hline
			Fusion & An Autodesk Cloud-based 3D CAD/CAM tool/product \\ \hline
			Task & In the context of Fusion, a function or operation that can be performed in Fusion \\ \hline
			Plugin & Software that adds specific new functionlity to another piece of software \\ \hline
			User & A person that interacts with Kora or Fusion depending on the context \\ \hline
			Workflow & A sequence of related tasks \\ \hline
		\end{tabular}
	\end{table}

\section{Design Stakeholders and Concerns}


%%%%%%%%%%%%%%%%%%%%%%%%%%%%%%%%%%%%%%%%%%%%%%%%%%%%%%%%%%%%%%%%%%%%%%%%%%%%%%%%%%%%%%%%%%%%%%%%%%%%%%%%%%% 
%                                           ***NOTES***
%
% Addressed Design Concerns:
%    1) Should specify by ID each of the concerns from the "Design Stakeholders and Concerns"
%       that are being addressed or discussed in that particular viewpoint.
%
% Design Elements: 
%    1) If element is not previously defined in other viewpoint, then offer definition. 
%       Otherwise reference definition in other viewpoint (e.g. "See 4.5.2")
%
% Design Rationale:
%    1) There won't be a single section for design rationale. As per the IEEE standards doc,
%       the design rationale should be justification for decisions made in the desing and
%       it doesn't need a dedicated section. Just make sure to include justification for 
%       why we chose to do something in a certain way (e.g. "Use of the Mediator design pattern
%       allows the application to...")
%%%%%%%%%%%%%%%%%%%%%%%%%%%%%%%%%%%%%%%%%%%%%%%%%%%%%%%%%%%%%%%%%%%%%%%%%%%%%%%%%%%%%%%%%%%%%%%%%%%%%%%%%%% 
\section{Design Viewpoint: Composition}
    \subsection{Addressed Design Concerns}
        \begin{itemize}
            \item \designConcernRef[Optional-Arg]{Concern-Identifier}
        \end{itemize}

    \subsection{Design Elements} 
        \designElementDef{Element-Name}{Element-Type}{Element-Purpose}{Element-Author}
        \designElementDef{Speech-to-Intent Worker Module}
                         {Module}
                         {To encapsulate the natural language processing functionality of \botname.}
                         {Element-Author}
    \subsection{Design View: Modules}


\section{Design Viewpoint: Information}
    \subsection{Addressed Design Concerns}
        \begin{itemize}
            \item
        \end{itemize}

    \subsection{Design Elements} 

    \subsection{Design View: }


\section{Design Viewpoint: Patterns}
    \subsection{Addressed Design Concerns}
        \begin{itemize}
            \item
        \end{itemize}

    \subsection{Design Elements}

    \subsection{Design View: }


\section{Design Viewpoint: Interfaces}
    \subsection{Addressed Design Concerns}
        \begin{itemize}
            \item
        \end{itemize}

    \subsection{Design Elements}

    \subsection{Design View: }


\section{Design Viewpoint: Interactions}
    \subsection{Addressed Design Concerns}
        \begin{itemize}
            \item
        \end{itemize}

    \subsection{Design Elements}

    \subsection{Design View: }


\section{Design Viewpoint: State Dynamics}
    \subsection{Addressed Design Concerns}
        \begin{itemize}
            \item
        \end{itemize}

    \subsection{Design Elements}

    \subsection{Design View: }
bibliography{requirementsBib} 
%\bibliographystyle{ieeetr}

\end{document}
