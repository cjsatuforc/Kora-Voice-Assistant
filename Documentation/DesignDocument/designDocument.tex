\documentclass[onecolumn, draftclsnofoot,10pt, compsoc]{IEEEtran}
\usepackage{graphicx}
\usepackage[section]{placeins}
\usepackage{url}
\usepackage{setspace}
 
\usepackage{alltt}                                           
\usepackage{float}
\usepackage{color}
\usepackage{url}

\usepackage{geometry}
\geometry{textheight=9.5in, textwidth=7in}
\setlength\parindent{0pt}

\usepackage{xspace}
\usepackage{pgfgantt}
\usepackage{subcaption}

% 1. Fill in these details
\def \CapstoneTeamName{\textbf{Insert Team Name Here} }
\def \CapstoneTeamNumber{8}
\def \GroupMemberOne{James Stallkamp}
\def \GroupMemberTwo{Jeremy Fischer}
\def \GroupMemberThree{Austin Row}
\def \CapstoneProjectName{Kora}
\def \CapstoneSponsorCompany{Autodesk}
\def \CapstoneSponsorPerson{Patti Vrobel}
\def \botname{Kora\xspace}

% 2. Uncomment the appropriate line below so that the document type works
\def \DocType{		%Problem Statement
				%Requirements Document
				%Technology Review
				Design Document
				%Progress Report
				}

\newcommand{\designConcernDef}[3]{
    \subsection{#1}
        \begin{tabular}[t]{r p{6in}}
            Stakeholder(s): & #2 \\
            Concern: & #3 \\
        \end{tabular}
}
\newcommand{\designConcernRef}[2][]{
    #2 #1
}
\newcommand{\designElementDef}[4]{
    \subsubsection{#1}
    \begin{tabular}[t]{r p{6in}}
        Type: & #2 \\
        Purpose: & #3 \\
        Author: & #4 \\
    \end{tabular}
}
\newcommand{\designElementRef}[2]{
    \subsubsection{#1}
    \begin{tabular}[t]{r p{6in}}
        See #2 & \\ %#2 should be element identifier (section where it's defined or ID that can be used to find it)
    \end{tabular}
}
\newcommand{\NameSigPair}[1]{\par
\makebox[2.75in][r]{#1} \hfil 	\makebox[3.25in]{\makebox[2.25in]{\hrulefill} \hfill		\makebox[.75in]{\hrulefill}}
\par\vspace{-12pt} \textit{\tiny\noindent
\makebox[2.75in]{} \hfil		\makebox[3.25in]{\makebox[2.25in][r]{Signature} \hfill	\makebox[.75in][r]{Date}}}}
% 3. If the document is not to be signed, uncomment the RENEWcommand below
\renewcommand{\NameSigPair}[1]{#1}

%%%%%%%%%%%%%%%%%%%%%%%%%%%%%%%%%%%%%%%
\begin{document}
\begin{titlepage}
    \pagenumbering{gobble}
    \begin{singlespace}
    	\includegraphics[height=4cm]{coe_v_spot1}
        %\hfill 
        % 4. If you have a logo, use this includegraphics command to put it on the coversheet.
        \par\vspace{.2in}
        \centering
        \scshape{
            \huge CS Capstone \DocType \par
            {\large\today}\par
            \vspace{.5in}
            \textbf{\Huge\CapstoneProjectName}\par
            \vfill
            {\large Prepared for}\par
            \Huge \CapstoneSponsorCompany\par
            \vspace{5pt}
            {\Large\NameSigPair{\CapstoneSponsorPerson}\par}
            {\large Prepared by }\par
            Group\CapstoneTeamNumber\par
            % 5. comment out the line below this one if you do not wish to name your team
            %\CapstoneTeamName\par 
            \vspace{5pt}
            {\Large
                \NameSigPair{\GroupMemberOne}\par
                \NameSigPair{\GroupMemberTwo}\par
                \NameSigPair{\GroupMemberThree}\par
            }
            \vspace{20pt}
        }
        \begin{abstract}

        \end{abstract}     
    \end{singlespace}
\end{titlepage}
\newpage
\pagenumbering{arabic}
\tableofcontents
% 7. uncomment this (if applicable). Consider adding a page break.
%\listoffigures
%\listoftables
\clearpage

% 8. now you write!


\section{Frontspiece}
	\subsection{Date of Issue and Status}
		This document was issued November 30, 2017 and is the first draft of the design document.
	
	\subsection{Issuing Organization}
		This document has been issued by Oregon State University and Autodesk via the senior design capstone class.
	
	\subsection{Authorship}
		Jeremy Fischer, Austin Row, and James Stallkamp are the authors of this document and the developers of Kora.
		
	\subsection{Change History}
		\begin{table}[H]
			\centering
			\caption{Change History}
			\label{my-label}
			\begin{tabular}{|l|l|}
				\hline
				\textbf{Date}     & \textbf{Change Description}   \\ \hline
				November 30, 2017 & {First design document draft} \\ \hline
			\end{tabular}
		\end{table}

\section{Introduction}
	\subsection{Purpose}
		The purpose of this design document is to outline how Kora's workflows will be completed and connected.
		More generally, this document describes how the client's requirements will be met.
		Kora's developers will use this document as a roadmap during implementation.

	\subsection{Scope}
		This document focuses on the relationships between Kora's components and their individual processes, and how they work together to satisfy the project's requirements.
	
    \subsection{Context}
        TODO: FILL THIS IN

	\subsection{Summary}
		Kora will be a speech-based virtual assistant for Fusion that lets users perform any one of a subset of tasks within the product, such as saving a document or opening a menu, by verbally instructing it to perform the task.
		Workflows in Fusion that are not suited for handling by a voice interface will not be supported by Kora.
		As a stretch goal, Kora will be capable of questioning the user and using responses to predict and automatically assist with future user behavior.
		It will be a plugin that is bundled with Fusion and will be part of the product's standard download. 
		
		Kora will offer users a tool that decreases the time required to achieve their goals within Fusion by offering an interface that runs in parallel with and complements the keyboard and mouse.
		If the stretch goal is achieved, Kora will further increase productivity by learning to predict and automate specific workflows within the product.
    
\section{References}


\section{Glossary}
	\begin{table}[H]
		\centering
		\caption{Glossary}
		\label{my-label}
		\begin{tabular}{|l|l|}
			\hline
			\textbf{Term} & \textbf{Definition} \\ \hline
			Kora & The virtual assistant that is the focus of this project \\ \hline
			NLP & Natural Language Processing \\ \hline
			API & Application Programing Interface \\ \hline
			CAD & Computer Aided Design \\ \hline
			CAM & Computer Aided Manufacturing \\ \hline
			Fusion & An Autodesk Cloud-based 3D CAD/CAM tool/product \\ \hline
			Task & In the context of Fusion, a function or operation that can be performed in Fusion \\ \hline
			Plugin & Software that adds specific new functionlity to another piece of software \\ \hline
			User & A person that interacts with Kora or Fusion depending on the context \\ \hline
			Workflow & A sequence of related tasks \\ \hline
		\end{tabular}
	\end{table}

\section{Design Stakeholders and Concerns}
    \designConcernDef{User-Application Interface}{Client, Developers}{What interfaces will exist for users to communicate to the application and for the application to communicate to users?}
    \designConcernDef{Internal Interfaces}{Developers}{What internal interfaces will exist in the software system?}
    \designConcernDef{Component Interactions}{Developers}{Which components of the software system will interact?}
    \designConcernDef{Component Interaction Resposibilities}{Developers}{What are the reponsibilities of each component of the software system in the context of interactions?}
	\designConcernDef{Architecture}{Developers}{How is the system structured?}

%%%%%%%%%%%%%%%%%%%%%%%%%%%%%%%%%%%%%%%%%%%%%%%%%%%%%%%%%%%%%%%%%%%%%%%%%%%%%%%%%%%%%%%%%%%%%%%%%%%%%%%%%%% 
%                                           ***NOTES***
%
% Addressed Design Concerns:
%    1) Should specify by ID each of the concerns from the "Design Stakeholders and Concerns"
%       that are being addressed or discussed in that particular viewpoint.
%
% Design Elements: 
%    1) If element is not previously defined in other viewpoint, then offer definition. 
%       Otherwise reference definition in other viewpoint (e.g. "See 4.5.2")
%
% Design Rationale:
%    1) There won't be a single section for design rationale. As per the IEEE standards doc,
%       the design rationale should be justification for decisions made in the desing and
%       it doesn't need a dedicated section. Just make sure to include justification for 
%       why we chose to do something in a certain way (e.g. "Use of the Mediator design pattern
%       allows the application to...")
%%%%%%%%%%%%%%%%%%%%%%%%%%%%%%%%%%%%%%%%%%%%%%%%%%%%%%%%%%%%%%%%%%%%%%%%%%%%%%%%%%%%%%%%%%%%%%%%%%%%%%%%%%% 
\section{Design Viewpoint: Composition}
    \subsection{Addressed Design Concerns}
        \begin{itemize}
            \item \designConcernRef[Optional-Arg]{Module Composition}
        \end{itemize}

    \subsection{Design Elements} 
        \designElementDef{Manager Module}
                         {Module}
                         {Manages what state \botname is in and executes other modules.}
                         {James Stallkamp}
        \designElementDef{User Interface Module}
                         {Module}
                         {Provides input to \botname and expresses the state of \botname to the user.}
                         {James Stallkamp}
        \designElementDef{Speech-To-Intent Module}
                         {Module}
                         {Anaylzes speech input and produces an intent.json object.}
                         {James Stallkamp}
        \designElementDef{Logger Module}
                         {Module}
                         {Stores runtime and contextual information to be used for training \botname.}
                         {James Stallkamp}
        \designElementDef{Fusion API Module}
                         {Module}
                         {Translates intent into Fusion API commands and executes them.}
                         {James Stallkamp}
        \designElementDef{Voice Synthesizer Module}
                         {Module}
                         {Synthesizes an audio output from a given text input.}
                         {James Stallkamp}
        \designElementDef{Machine Learning Module}
                         {Module}
                         {Trains \botname to become a more powerful assistant.}
                         {James Stallkamp}
    \subsection{Design View: Modules}
		\botname is composed of seven primary modules.
		The first module is the manager module, this module is responsible for coordinating all other modules.
		The master module will contain definitions for functions needed to process data objects, other modules will inherit this from master.
		The next module is the User interface and is the module responsible for all interaction with the user.
		This module will collect input and communicate it to master as well output regular feedback to the user.
		The speech-to-intent module will take in audio input and ouput a json object containing information on the spoken input.
		This intent module will construct the json object and return it to master.
		The Logger module will initialize a persistable data object containing runtime information from \botname.
		Data logged will be used to help train and improve \botname.
		The next module is the Fusion API, this module will handle executing Fusion commands.
		The Fusion module will translate information in the json data object to constuct and execute a Fusion comman.
		In order for \botname to output speed to the user it will need a voice synthesizer module.
		This module will recieve text input and produce an audio output that can be played to the user.
		The last module is the machine learning module, this module is responsible for training \botname to recognize patters and improve functionality.

\section{Design Viewpoint: Information}
    \subsection{Addressed Design Concerns}
        \begin{itemize}
            \item
        \end{itemize}

    \subsection{Design Elements} 

    \subsection{Design View: }
		
		

\section{Design Viewpoint: Patterns}
    \subsection{Addressed Design Concerns}
        \begin{itemize}
            \item \designConcernRef[Optional-Arg]{Architecture}
        \end{itemize}

    \subsection{Design Elements}
		\designElementDef{Mediator Framework}
						 {System Framework}
						 {\botname has a simple mediator that coordinates all interactions between all other components.}
						 {James Stallkamp}
		\designElementDef{Worker Class}
						 {Class}
						 {A inheritable class providing functionality to handle the json object}
						 {James Stallkamp}				 
    \subsection{Design View: }
		\botname is structured in a mediator format. 
		\botname will have a master module that coordinates interactions between itself and all other modules.
		This master module will provide basic funtionality needed to process data objects, other modules will inherit this from master. 


\section{Design Viewpoint: Interfaces}
    \subsection{Addressed Design Concerns}
        \begin{itemize}
            \item \designConcernRef[User-Application Interface]{ID}
            \item \designConcernRef[Internal Interfaces]{ID}
        \end{itemize}

    \subsection{Design Elements}
        \designElementDef{Element-Name}{Element-Type}{Element-Purpose}{Element-Author}
        \designElementDef{UI Module Interface}{Internal Interface}{Defines rules governing interactions with the UI module.}{Austin Row}
        \designElementDef{Speech-to-Intent Module Interface}{Internal Interface}{Defines rules governing interactions with the Speech-to-Intent module.}{Austin Row}
        \designElementDef{Logging Module Interface}{Internal Interface}{Defines rules governing interactions with the Logging module.}{Austin Row}
        \designElementDef{Text-to-Speech Module Interface}{Internal Interface}{Defines rules governing interactions with the Text-to-Speech module.}{Austin Row}
        \designElementDef{Fusion Module Interface}{Internal Interface}{Defines rules governing interactions with the Fusion module.}{Austin Row}
        \designElementDef{Action Prediction Module Interface}{Internal Interface}{Defines rules governing interactions with the Action Prediction module.}{Austin Row}
        \designElementDef{Runtime Data Store Interface}{Internal Interface}{Defines rules governing interactions with the global Runtime Data Store.}{Austin Row}
    \subsection{Design View: Module Interfaces}
        The interfaces to each module are driven by a method, \textit{handle}, that takes a JSON variable, \textit{data}, as an argument.
        The required data transmitted in \textit{data} is specific to each module. 
        The following interface definitions describe the required contents of \textit{data} for each module.

        \subsubsection{UI Module Interface}
            \begin{tabular}[t]{r p{6in}}
                INPUT: & \\
                OUTPUT: & \\
            \end{tabular}

        \subsubsection{Speech-to-Intent Module Interface}
            \begin{tabular}[t]{r p{6in}}
                INPUT: &  \{ dataIDs: \}\\
                OUTPUT: & This module returns a \\
            \end{tabular}

\section{Design Viewpoint: Interactions}
    \subsection{Addressed Design Concerns}
        \begin{itemize}
            \item \designConcernRef[Component Interactions]{ID}
            \item \designConcernRef[Component Interaction Responsibilities]{ID}
        \end{itemize}

    \subsection{Design Elements}
        \designElementDef{Element-Name}{Element-Type}{Element-Purpose}{Element-Author}

    \subsection{Design View: }


\section{Design Viewpoint: State Dynamics}
    \subsection{Addressed Design Concerns}
        \begin{itemize}
            \item
        \end{itemize}

    \subsection{Design Elements}

    \subsection{Design View: }
bibliography{requirementsBib} 
%\bibliographystyle{ieeetr}

\end{document}
