\documentclass[onecolumn, draftclsnofoot,10pt, compsoc]{IEEEtran}
\usepackage{graphicx}
\usepackage[section]{placeins}
\usepackage{url}
\usepackage{setspace}
 
\usepackage{alltt}                                           
\usepackage{float}
\usepackage{color}
\usepackage{url}

\usepackage{geometry}
\geometry{textheight=9.5in, textwidth=7in}
\setlength\parindent{0pt}

\usepackage{tabu}
\usepackage{longtable}

\usepackage{xspace}
\usepackage{pgfgantt}
\usepackage{subcaption}

% 1. Fill in these details
\def \CapstoneTeamNumber{8}
\def \GroupMemberOne{Austin Row}
\def \GroupMemberTwo{Jeremy Fischer}
\def \GroupMemberThree{James Stallkamp}
\def \CapstoneProjectName{Kora}
\def \CapstoneSponsorCompany{Autodesk}
\def \CapstoneSponsorPerson{Anand Karyekar}
\def \botname{Kora\xspace}

% 2. Uncomment the appropriate line below so that the document type works
\def \DocType{		%Problem Statement
				%Requirements Document
				%Technology Review
				%Design Document
				Winter Progress Report
				}
			
\newcommand{\NameSigPair}[1]{\par
\makebox[2.75in][r]{#1} \hfil 	\makebox[3.25in]{\makebox[2.25in]{\hrulefill} \hfill		\makebox[.75in]{\hrulefill}}
\par\vspace{-12pt} \textit{\tiny\noindent
\makebox[2.75in]{} \hfil		\makebox[3.25in]{\makebox[2.25in][r]{Signature} \hfill	\makebox[.75in][r]{Date}}}}
% 3. If the document is not to be signed, uncomment the RENEWcommand below
\renewcommand{\NameSigPair}[1]{#1}

%%%%%%%%%%%%%%%%%%%%%%%%%%%%%%%%%%%%%%%
\begin{document}
\begin{titlepage}
    \pagenumbering{gobble}
    \begin{singlespace}
    	\includegraphics[height=4cm]{coe_v_spot1}
        %\hfill 
        % 4. If you have a logo, use this includegraphics command to put it on the coversheet.
        \par\vspace{.2in}
        \centering
        \scshape{
            \huge CS Capstone \DocType \par
            {\large\today}\par
            \vspace{.5in}
            \textbf{\Huge\CapstoneProjectName}\par
            \vfill
            {\large Prepared for}\par
            \Huge \CapstoneSponsorCompany\par
            \vspace{5pt}
            {\Large\NameSigPair{\CapstoneSponsorPerson}\par}
            {\large Prepared by }\par
            Group\CapstoneTeamNumber\par
            % 5. comment out the line below this one if you do not wish to name your team
            %\CapstoneTeamName\par 
            \vspace{5pt}
            {\Large
                \NameSigPair{\GroupMemberOne}\par
                \NameSigPair{\GroupMemberTwo}\par
                \NameSigPair{\GroupMemberThree}\par
            }
            \vspace{20pt}
        }
        \begin{abstract}
			This document outlines the production progress of Kora from December 2017 through the middle of February 2018, including obstacles and steps moving forward.
        \end{abstract}     
    \end{singlespace}
\end{titlepage}
\newpage
\pagenumbering{arabic}
\tableofcontents
% 7. uncomment this (if applicable). Consider adding a page break.
%\listoffigures
%\listoftables
\clearpage

% 8. now you write!

\section{Purpose}
	Kora is a proof of concept project that will integrate a natural language processing library into Autodesk's 3-D computer aided design software, Fusion.
	Kora will be a speech-based virtual assistant for Fusion that lets users perform a subset of tasks within the product, such as saving a document or opening a menu, by verbally instructing it to perform the task.
	As a stretch goal, Kora will be capable of questioning the user and using responses to predict and automatically assist with future user behavior.
	
	Kora will offer users a tool that decreases the time required to achieve their goals within Fusion by offering an interface that runs in parallel with and complements the keyboard and mouse.
	If the stretch goal is achieved, Kora will further increase productivity by learning to automate specific workflows within the product.

\section{Goals}
	\subsection{Main Goals}
		\begin{itemize}
			 \item{
			 	Kora will allow the user to perform tasks in Fusion via a voice interface.
				Each voice command given by the user will be mapped to a specific task which will then be performed in the open Fusion design.}
			\item{
				All interactions with Kora will be logged for debugging and future development purposes.}
		\end{itemize}
	\subsection{Stretch Goals}
		\begin{itemize}
			\item{
				Kora will periodically ask the user questions regarding why they have performed specific tasks.
				The questions asked by Kora will pertain to specific predefined contexts.}
			\item{
			Kora will record user responses and save a transcript along with contextual information.}
			\item{
				User responses to interview questions will be used to try to predict and assist with future user actions.}
		\end{itemize}
\section{Project Progress}
	%TODO%
	

\section{Obstacles}
	%TODO%
	

\section{Moving Forward}
	%TODO%


\section{Photos}
	%TODO%
	The assignment page said: \textit{includes images of your project -- screen shots, photos, whatever is appropriate.}
	So, maybe we could add a picture of the tool Add-In drop down that says "Activate/Deactive Kora", and maybe a Fusion message box popping up with something relevant on it.

\end{document}
